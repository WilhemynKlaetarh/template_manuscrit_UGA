% %%%%%%%%%%%%%%%%%%%%%%%%%%%%%%%%%%%%%%%%%%%%%%%%%%%%%%%%%%%%%%%%%%%%%%%%%%
% %                            PAQUETS USUELS                              %
% %                                                   %
% %%%%%%%%%%%%%%%%%%%%%%%%%%%%%%%%%%%%%%%%%%%%%%%%%%%%%%%%%%%%%%%%%%%%%%%%%%


\usepackage[T1]{fontenc}
\usepackage[utf8]{inputenc}
%\usepackage[francais]{babel}

\usepackage{fontspec}
\usepackage{libertine}
%\setmainfont[Ligatures={Historic}]{Linux Libertine O}
\usepackage{amssymb}
\let\mathbbalt\mathbb
\usepackage{unicode-math}
\setmathfont{Latin Modern Math}%[version=lm]
\DeclareMathAlphabet{\mathcal}{OMS}{cmsy}{m}{n}
\everymath{\displaystyle}

\usepackage[french]{babel}  % francais = french babel
\usepackage{meta-donnees} % page de garde
\usepackage{meta-donnees2} % page de 4ième de couv
\usepackage{verbatim}

\usepackage{array}	
\usepackage{color}
\usepackage{cite}
\usepackage{xfrac}
%\usepackage{mathptmx}
\usepackage{enumitem}
\usepackage[a]{esvect} % pour avoir des flèches de vecteur plus stylées
\usepackage[squaren,Gray]{SIunits}
\usepackage{sistyle}
\usepackage{eurosym}  %pour obtenir le symbole Euro
%\usepackage{gensymb} %pour obtenir le symbole \degree
\usepackage{calligra}
\usepackage{multicol}

\usepackage{tikz} % TikZ est utilisé pour la page de garde


\usepackage{animate}

\usepackage{blindtext}



%\usepackage{fourier} 
%\usepackage[no-math]{fontspec}
%\usepackage{complexity}

%\usepackage{lmodern}
%\usepackage{unicode-math}
%\usepackage{xfrac,unicode-math}

%\setmathfont{Latin Modern Math}%[version=lm]


%\setmathfont{Times New Roman}

%\usepackage[math]{blindtext}

% Font fourier
%
% Only use the math font of mathpazo
%\let\temp\rmdefault
%\usepackage{lmodern}
%\let\rmdefault\temp


%%%%%%%%%%%%%%%%%%%%%%%%%%%%%%%%%%%%%%%%%%%%%%%%%%%%%%%%%%%%%%%%%%%%%%%%%%
%%%%%           Packages pour les entetes et pied de page           %%%%%%
%%%%%%%%%%%%%%%%%%%%%%%%%%%%%%%%%%%%%%%%%%%%%%%%%%%%%%%%%%%%%%%%%%%%%%%%%%

%\usepackage{picins}
\usepackage{fancyhdr}
%\usepackage{psboxit}  %%% A Inserer avant babel !!!! 
\usepackage{pifont}


%%%%%%%%%%%%%%%%%%%%%%%%%%%%%%%%%%%%%%%%%%%%%%%%%%%%%%%%%%%%%%%%%%%%%%%%%%
%%%%%                   Packages et couleurs perso                  %%%%%%
%%%%%%%%%%%%%%%%%%%%%%%%%%%%%%%%%%%%%%%%%%%%%%%%%%%%%%%%%%%%%%%%%%%%%%%%%%

\usepackage{xcolor}  %%% Incompatibilité avec \usepackage{colortbl} ??????

% Logo UGA pour la page de garde
\definecolor{orangeUni}{HTML}{dd6437}
\definecolor{bleuUni}{HTML}{242b45}

%%% Définition des couleurs personnelles
\definecolor{BleuCyan}{RGB}{0,190,190}  
\definecolor{RoseRose}{RGB}{238,44,44}
\definecolor{VertVert}{RGB}{10,255,118}
\definecolor{Anthracite}{RGB}{91,124,151}
\definecolor{GrisPasTropClair}{RGB}{83,135,135}
\definecolor{BleuPetrole}{RGB}{0, 0 ,205}
\definecolor{BleuClair}{RGB}{234, 255 ,255}
\definecolor{Bleu1}{RGB}{26, 64 ,145}
\definecolor{Rouge1}{RGB}{215, 19 ,24}
\definecolor{myblue}{rgb}{.8, .8, 1}
\definecolor{violet1}{rgb}{0.78,0.53,0.97}
\definecolor{myblue2}{rgb}{0,0.41,0.54} % dark blue
\definecolor{myred}{RGB}{192,0,0} % dark red
\definecolor{mygreen2}{RGB}{0,120,0} % dark green
\definecolor{myblue3}{HTML}{1a4091} % dark blue 	
\definecolor{couleur_marie}{RGB}{0,102,204} % dark green




%\newcommand*\maboite[1]{%
%\fcolorbox{GrisPasTropClair}{BleuClair}{\hspace{1em}#1\hspace{1em}}}
%\newcommand{\parttoccolor}{blue}
%\newcommand{\chaptertoccolor}{red}
%\newcommand{\sectiontoccolor}{green!70!black}

\usepackage{tocloft}
\renewcommand{\cftpartafterpnum}{%
  \strut\par\nopagebreak%  % begin new paragraph and prevent page break
  \vskip0.25ex\hrule%      % draw horizontal rule after small vertical skip
}

%\usepackage{geometry}
\usepackage[printonlyused,withpage]{acronym}
%\usepackage[withpage]{acronym} % Version sans le numéro de page dans la table (cf. documentation du package)


%%%%%%%%%%%%%%%%%%%%%%%%%%%%%%%%%%%%%%%%%%%%%%%%%%%%%%%%%%%%%%%%%%%%%%%%%%
%%%%%            Packages pour les captions optimisés               %%%%%%
%%%%%%%%%%%%%%%%%%%%%%%%%%%%%%%%%%%%%%%%%%%%%%%%%%%%%%%%%%%%%%%%%%%%%%%%%%
\usepackage[font=small,font={it}]{caption} % \usepackage[small,hang]{caption2} apparement cpation2 est obsolette
\usepackage{subcaption}
%\captionsetup[table]{position=bottom}
%\renewcommand{\captionfont}{\it \small}
%\renewcommand{\captionlabelfont}{\it \bf \small}
% \renewcommand{\captionlabeldelim}{ :}  % Ne marche qu'avec caption2



%%%%%%%%%%%%%%%%%%%%%%%%%%%%%%%%%%%%%%%%%%%%%%%%%%%%%%%%%%%%%%%%%%%%%%%%%%
%%%%%           Packages pour les titres de chapitres               %%%%%%
%%%%%%%%%%%%%%%%%%%%%%%%%%%%%%%%%%%%%%%%%%%%%%%%%%%%%%%%%%%%%%%%%%%%%%%%%%
%\usepackage[Bjornstrup]{fncychap}
\usepackage[libertine]{quotchap}

\definecolor{chaptergrey}{HTML}{FF0000} % Changer la couleur des numéros de chapitre

\usepackage{lettrine}    %Lettrine exemple: \lettrine[lines=2]{L}{orem ipsum}
\usepackage[francais]{minitoc}		% Pour ajouter une table des matières à chaque chapitre
\setcounter{minitocdepth}{2}

%\renewcommand{\contentsname}{\hfill\fontsize{20pt}{20pt}\selectfont\textnormal{Table des matières}}
\renewcommand{\cfttoctitlefont}{\hfill\fontsize{20pt}{20pt}\selectfont\textnormal}
\renewcommand{\cftloftitlefont}{\hfill\fontsize{20pt}{20pt}\selectfont\textnormal}
\renewcommand{\cftlottitlefont}{\hfill\fontsize{20pt}{20pt}\selectfont\textnormal}

\usepackage{eso-pic}

% Commande pour ajouter \thepart en arrière-plan
\newcommand{\PartBackground}{%
  \AddToShipoutPicture*{%
    \put(0,0){%
      \parbox[b][\paperheight]{\paperwidth}{%
        \vfill
        \centering
        \fontsize{15cm}{15cm}\selectfont
        \textcolor{VertVert!30}{\hphantom{-}\thepart}% % Changer la couleur par celle de son choix
        \vfill
      }%
    }%
  }%
}

% Redéfinition de la commande \part pour inclure \thepart en arrière-plan
\let\oldpart\part
\renewcommand{\part}[1]{%
  \clearpage
  %\refstepcounter{part}% Avance le compteur de \part
  \PartBackground% Affiche \thepart en arrière-plan
  \oldpart{#1}% Titre de la partie tel que défini
  \AddToShipoutPicture{} % Réinitialisation de l'arrière-plan
}
\newcommand{\partfin}[1]{%
 \clearpage
  %\refstepcounter{part}% Avance le compteur de \part
  \PartBackground% Affiche \thepart en arrière-plan
  \oldpart*{#1}% Titre de la partie tel que défini
  \AddToShipoutPicture{} % Réinitialisation de l'arrière-plan
}

%\renewcommand{\thesubsubsection}{\arabic{chapter}.\arabic{section}.\arabic{subsection}.\arabic{subsubsection}}

%%%%%%%%%%%%%%%%%%%%%%%%%%%%%%%%%%%%%%%%%%%%%%%%%%%%%%%%%%%%%%%%%%%%%%%%%%
%%%%%                Packages pour les figures                      %%%%%%
%%%%%%%%%%%%%%%%%%%%%%%%%%%%%%%%%%%%%%%%%%%%%%%%%%%%%%%%%%%%%%%%%%%%%%%%%%
\usepackage{epsfig}
\usepackage{wrapfig}  %%% Inserer du texte a droite ou à gauche de l'image
%\usepackage{picins}  %%% Inserer du texte a droite ou à gauche de l'image
\usepackage{float}
\usepackage{graphicx}
%\usepackage{subfigure}
\usepackage{subcaption}
\usepackage[inkscape=newer]{svg} 

%%%%%%%%%%%%%%%%%%%%%%%%%%%%%%%%%%%%%%%%%%%%%%%%%%%%%%%%%%%%%%%%%%%%%%%%%%
%%%%%                  Packages pour les tableaux                   %%%%%%
%%%%%%%%%%%%%%%%%%%%%%%%%%%%%%%%%%%%%%%%%%%%%%%%%%%%%%%%%%%%%%%%%%%%%%%%%%
\usepackage{array}
\usepackage{textcomp}
\usepackage{booktabs}
\usepackage{colortbl}  %%% Couleurs de cellule de tableau
\usepackage{longtable}
\usepackage{lscape}    %%% Rotation des tableaux
\usepackage{multirow}
\usepackage{tabularx}
%\usepackage{slashbox}
\usepackage{pifont}
%\usepackage[table]{xcolor}

\usepackage{colortbl}

\arrayrulecolor{myblue3}
\let\oldtabular=\tabular 
\def\tabular{\small\oldtabular}

%%%%%%%%%%%%%%%%%%%%%%%%%%%%%%%%%%%%%%%%%%%%%%%%%%%%%%%%%%%%%%%%%%%%%%%%%%
%%%%%                  Packages pour les maths                  %%%%%%
%%%%%%%%%%%%%%%%%%%%%%%%%%%%%%%%%%%%%%%%%%%%%%%%%%%%%%%%%%%%%%%%%%%%%%%%%%
\usepackage{amsmath}
\usepackage{bm}
\usepackage[version=4]{mhchem}
\usepackage{stmaryrd}

%\usepackage{stackengine}
%\usepackage{empheq} % Pour encadrer les equations
%pour utiliser plusieurs fichiers bibteX
%\usepackage{biblist}



%\usepackage{fourier}
%\usepackage{SIunits}


%%%%%%%%%%%%%%%%%%%%%%%%%%%%%%%%%%%%%%%%%%%%%%%%%%%%%%%%%%%%%%%%%%%%%%%%%%
%%%%%              Packages pour les liens hypertexte               %%%%%%
%%%%%%%%%%%%%%%%%%%%%%%%%%%%%%%%%%%%%%%%%%%%%%%%%%%%%%%%%%%%%%%%%%%%%%%%%%
\usepackage{hyperref} %%%  Packages pour les liens hypertexte a mettre après tous les autres packages
% Pour changer les couleurs des liens dans le Manuscrit : les liens URL, les liens des références biblio, les liens vers les chapitres et les sections
 \hypersetup{
 colorlinks=true,
 linkcolor=BleuCyan,
 citecolor=RoseRose,
 urlcolor=RoseRose,
 filecolor=VertVert
 }
 \urlstyle{rm}
%  hyperindex=true,